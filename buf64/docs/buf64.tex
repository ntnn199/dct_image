\documentclass[11pt]{article}

\usepackage{times}
\usepackage{epsf}
\usepackage{epsfig}
\usepackage{amsmath, alltt, amssymb, xspace}
\usepackage{wrapfig}
\usepackage{fancyhdr}
\usepackage{url}
\usepackage{verbatim}
\usepackage{fancyvrb}
\usepackage{float}

\usepackage{subfigure}
\usepackage{cite}
\usepackage{hyperref}
\hypersetup{%
    pdfborder = {0 0 0}
}
\topmargin      -0.50in  % distance to headers
\oddsidemargin  0.0in
\evensidemargin 0.0in
\textwidth      6.5in
\textheight     8.9in 


%\centerfigcaptionstrue

%\def\baselinestretch{0.95}


\newcommand\discuss[1]{\{\textbf{Discuss:} \textit{#1}\}}
%\newcommand\todo[1]{\vspace{0.1in}\{\textbf{Todo:} \textit{#1}\}\vspace{0.1in}}
\newtheorem{problem}{Problem}[section]
%\newtheorem{theorem}{Theorem}
%\newtheorem{fact}{Fact}
\newtheorem{define}{Definition}[section]
%\newtheorem{analysis}{Analysis}
\newcommand\vspacenoindent{\vspace{0.1in} \noindent}

%\newenvironment{proof}{\noindent {\bf Proof}.}{\hspace*{\fill}~\mbox{\rule[0pt]{1.3ex}{1.3ex}}}
%\newcommand\todo[1]{\vspace{0.1in}\{\textbf{Todo:} \textit{#1}\}\vspace{0.1in}}

%\newcommand\reducespace{\vspace{-0.1in}}
% reduce the space between lines
%\def\baselinestretch{0.95}

\newcommand{\fixmefn}[1]{ \footnote{\sf\ \ \fbox{FIXME} #1} }
\newcommand{\todo}[1]{
\vspace{0.1in}
\fbox{\parbox{6in}{TODO: #1}}
\vspace{0.1in}
}

\newcommand{\mybox}[1]{
\vspace{0.2in}
\noindent
\fbox{\parbox{6.5in}{#1}}
\vspace{0.1in}
}


\newcounter{question}
\setcounter{question}{1}

\newcommand{\myquestion} {{\vspace{0.1in} \noindent \bf Question \arabic{question}:} \addtocounter{question}{1} \,}

\newcommand{\myproblem} {{\noindent \bf Problem \arabic{question}:} \addtocounter{question}{1} \,}


\newcommand{\copyrightnotice}[1]{
\vspace{0.1in}
\fbox{\parbox{6in}{
      This lab was developed for the Labtainer framework by the Naval Postgraduate
      School, Center for Cybersecurity and Cyber Operations under sponsorship from
      the DoD CySP program.  This work is in the public domain, and cannot be copyrighted.}}
\vspace{0.1in}
}


\newcommand{\idea}[1]{
\vspace{0.1in}
{\sf IDEA:\ \ \fbox{\parbox{5in}{#1}}}
\vspace{0.1in}
}

\newcommand{\questionblock}[1]{
\vspace{0.1in}
\fbox{\parbox{6in}{#1}}
\vspace{0.1in}
}


\newcommand{\argmax}[1]{
\begin{minipage}[t]{1.25cm}\parskip-1ex\begin{center}
argmax
#1
\end{center}\end{minipage}
\;
}

\newcommand{\bm}{\boldmath}
\newcommand  {\bx}    {\mbox{\boldmath $x$}}
\newcommand  {\by}    {\mbox{\boldmath $y$}}
\newcommand  {\br}    {\mbox{\boldmath $r$}}


\newcommand{\tstamp}{\today}   
%\rfoot[\fancyplain{\tstamp} {\tstamp}]  {\fancyplain{}{}}

\pagestyle{fancy}
\lhead{\bfseries Labtainers}
\chead{}
\rhead{\small \thepage}
\lfoot{}
\cfoot{}
\rfoot{}




\begin{document}

\begin{center}
{\LARGE Buffer Overflow: 64 Bit Application}
\vspace{0.1in}\\
\end{center}

\section{Overview}
The {\tt bufoverflow} lab introduced you to buffer overflow vulnerabilities and potential exploits of those
vulnerabilities.  That lab included a vulnerable program that ran as a 32-bit x86 application.  
This lab includes the very same vulnerable program source code, however it compiles and runs as a 64-bit application.

\subsection {Background}
The student is expected to have an understanding of the Linux command line, and some amount of low level programming.
It is expected that the student will have completed the {\tt bufoverflow} lab.

\section{Lab Environment}
\label{environment}
This lab runs in the Labtainer framework,
available at http://nps.edu/web/c3o/labtainers.
That site includes links to a pre-built virtual machine
that has Labtainers installed, however Labtainers can
be run on any Linux host that supports Docker containers.

From your labtainer-student directory start the lab using:
\begin{verbatim}
    labtainer buf64
\end{verbatim}
\noindent A link to this lab manual will be displayed.  

The home directory of the resulting computer contains the source code of the vulnerable program ({\tt stack.c}) and
a template for the program that constructs the malicious data file ({\tt exploit.c}).  It also includes files for generating
shell code.  

\section{Tasks}
You will modify the {\tt exploit.c} program such that it generates a malicious data file that will cause the
vulnerable program to enter a shell.  This lab does not require that you get a root shell -- an application shell
is sufficient.  And your exploit need only run with Address Space Layout Randomization disabled, with an executable
stack, and with stack protection disabled.

A learning objective of this lab is to appreciate some of the differences between 32-bit and 64-bit x86 applications,
and how those differences might affect vulnerabilities and exploits.


\subsection{Explore}
Review the differences between the files in this lab, and the files in the {\tt bufoverflow} lab.  Note the {\tt stack.c}
files are the same -- but with potentially different buffer sizes.  Look at the assembly in {\tt shell.c} and compare that
to the assembly comments to the object code found in the {\tt bufoverflow exploit.c} file.

Disable ASLR:
\begin{verbatim}
  sudo sysctl -w kernel.randomize_va_space=0
\end{verbatim}
\noindent and use the {\tt compile.sh} script to compile the C programs and assemble the shell.asm
Run the {\tt stack} program.  Run it in the debugger.  Explore.

\subsection{Shell code}
In this lab, you will need to update the {\tt exploit.c} program to include the shell code.  Observe that the 64-bit
shell code has been assembled into the shell.bin file.  You must figure out how to get that into your {\tt exploit.c}
program.  Note you have been provided with a Python script called {\tt hexit.py}, and that may be of use.  Look at the
file from the {\tt bufoverflow} lab as an example.

\subsection{Overwriting return address}
By now you should have observed that the {\tt stack} program crashes when it encounters a bad return address.
As in the {\tt bufoverflow} lab, you will want to control that return address value.

What might you want that return address value to be?  Consider the properties of such a value and how that might
affect the {\tt strcpy} function behavior.

\subsection{Get a shell}
Alter the {\tt exploit.c} so that it generates a badfile that causes the {\tt stack} application to give you  a shell.
Once you get a shell, cat the exploit.c file from within the shell:
\begin{verbatim}
   cat exploit.c
\end{verbatim}

\subsection{Follow on}
Think about how you might approach an exploit if the {\tt stack.c} program were compiled without disabling stack protection
and stack execution.

\section{Submission}
After finishing the lab, go to the terminal on your Linux system that was used to start the lab and type:
\begin{verbatim}
    stoplab 
\end{verbatim}
When you stop the lab, the system will display a path to the zipped lab results on your Linux system.  Provide that file to 
your instructor, e.g., via the Sakai site.

\copyrightnotice

\end{document}
